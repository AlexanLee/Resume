% !TEX TS-program = xelatex
% !TEX encoding = UTF-8 Unicode
% !Mode:: "TeX:UTF-8"

\documentclass{resume}
\usepackage{zh_CN-Adobefonts_external} % Simplified Chinese Support using external fonts (./fonts/zh_CN-Adobe/)
%\usepackage{zh_CN-Adobefonts_internal} % Simplified Chinese Support using system fonts
\usepackage{linespacing_fix} % disable extra space before next section
\usepackage{cite}

\begin{document}
\pagenumbering{gobble} % suppress displaying page number

\name{李见黎}

\basicInfo{
  \email{alexan.lee@pku.edu.cn} \textperiodcentered\ 
  \phone{(+86) 189-119-63820} \textperiodcentered\ 
  \linkedin[李见黎]{http://www.linkedin.com/in/alexan-见黎-lee-李-a9627257/}} 

\section{\faUsers\ 工作经历}
\datedsubsection{\textbf{猎豹移动} ,北京}{2017年5月 -- 至今}
\role{CTR、CVR预估}{广告算法}
\begin{onehalfspacing}
主要工作:猎豹移动广告变现算法维护、广告模型小时级别改造、PC模型构建、参加猎豹移动黑客马拉松模型组比赛等。
\begin{itemize}
  \item 接手并维护之前的模型数据流,切换集群等工作,保证原有模型的稳定运行。且在切换过程中详细重新梳理各个数据流和关系。使得数据流更加的清晰和可维护。 
  \item CTR模型小时级别的改造。顺利的完成改造,使点击率上升5\%左右。
  \item PC模型搭建,面对PC业务展现大,点击量小,数据十分稀疏的困难,采用采样的方式降低展现量,实验最后达到提升收入5\%至10\%的效果。
  \item 积极参加猎豹黑客马拉松模型组,根据文本预测twitter表情,取得了较好成绩。
  \item App数据分析,详细盘点广告数据和其他的数据与用户行为关系。
\end{itemize}
\end{onehalfspacing}

\datedsubsection{\textbf{猎豹移动},北京}{2016年3月 -- 至今}
\role{流量管理平台}{项目核心开发,项目负责人}
\begin{onehalfspacing}
主要工作:猎豹移动流量管理平台(SSP),作为公司的流量统一管理平台,需要兼顾多方的需求与业务(CFP,ADX,CSS,S2S,SDK等),需要面向诸多人群,主要包括流量方,运营,财务,数据等,有效的支撑公司变现业务的发展。参见:publishers.cmcm.com,manager.peg.cmcm.com
\begin{itemize}
  \item 参与系统的整体规划设计。系统基于Spring MVC、Shiro权限控制、Mybatis、MySQL、Druid、Active MQ等技术体系。
  \item 整个系统的后台开发。广告位管理、流量监控、Offer列表、API数据服务等核心功能的开发,自研的权限系统:包括功能权限,数据权限。
  \item 报表功能的开发,使用工厂方法等设计模式,可兼容多数据源,多样式输出的报表服务。
  \item 定时任务以及和其他系统交互服务的开发。
  \item 以Git作为项目管理工具,Jenkins作为构建工具。自己写的一键自动上线与线上部署脚本。
  \item 系统服务化改造,基于Spring Cloud进行服务化改造。 
  \item 流量监控专利一项:基于深度学习与移动广告数据的流量监控与评级系统。
\end{itemize}
\end{onehalfspacing}

\datedsubsection{\textbf{东软集团股份有限公司}}{2014年9月 -- 2015年10月}
\role{国税总局财务软件以及基建软件平台}{核心开发}
\begin{onehalfspacing}
工作内容:财务软件部分维护工作,以及不同类型的定制开发工作,基建软件的维护工作以及定制开发工作。
\begin{itemize}
  \item 工作挑战:沟通复杂,数据质量要求较高,业务逻辑复杂,适应范围广。
  \item 主要技术架构:java、weblogic、oracle、IBM MQ、linux架构。
\end{itemize}
\end{onehalfspacing}

\datedsubsection{\textbf{东软集团股份有限公司}}{2013年7月 -- 2014年8月}
\role{甘肃省地税数据分析平台}{核心开发}
\begin{onehalfspacing}
工作内容:针对多样化的税收相关数据,包括核心征管,建筑安装,发票信息,稽查法律,财务上报等数据,构建统一的数据分析平台。
\begin{itemize}
  \item 参与平台的整体设计,完成税收风险模块的设计、开发、维护工作,后期参与各种类型报表的设计开发工作。
  \item 主要挑战:数据源多样化,数据量大,且增量也较大,业务逻辑复杂,沟通多样。
  \item 主要技术架构,jsp、java、SAP、Oracle、data stage、linux、weblogic等。
  \item 其中进行了指标的特征抽取,利用聚类、决策树等机器学习方法。
  \item 项目获得了省级优秀信息化软件奖。
\end{itemize}
\end{onehalfspacing}


\datedsubsection{\textbf{东软集团股份有限公司}}{2012年07月 -- 2013年7月}
\role{青岛市地税税收风险管理系统}{核心开发}
\begin{onehalfspacing}
工作内容:针对税收征管数据,利用不同的风险指标,定时的监测数据,并推送和反馈给纳税人以及税务人员。
\begin{itemize}
  \item 税收具有业务逻辑复杂,数据量大,指标多样,数据要求严格,反馈及时等特点。
  \item 采用JSP、Struts、Spring、DRM(内部数据处理框架)、weblogic、Oracle、Linux/Aix/Solaris架构,利用多线程技术,分布式执行任务框架。
  \item 指标运算采用阈值法,切比雪夫法等多种计算模型。
  \item 经过专业人员甄别,再进行推送,对于推送流程逐步监控和反馈,及时得到结果。
\end{itemize}
\end{onehalfspacing}


\section{\faGraduationCap\  教育背景}
\datedsubsection{\textbf{北京大学}, 北京}{2016 -- 至今}
\textit{硕士}\ 软件工程, 预计 2019 年 3 月毕业。
 主要课程:软件体系结构与设计,算法设计与分析,云计算技术及应用,面向对象技术高级课程,软件工程过程与项目管理等。

\datedsubsection{\textbf{河北大学}, 河北}{2008 -- 2012}
\textit{学士}\ 信息与计算科学。
 主要课程:数学分析,高等代数,数据结构,操作系统,计算机组成原理,计算机网络,微机原理与接口技术等。



\section{\faCogs\ IT 技能}
% increase linespacing [parsep=0.5ex]
\begin{itemize}[parsep=0.5ex]
  \item 编程语言: JAVA、Python、SQL、Shell等。
  \item 平台: MAC、Linux。
  \item 软件设计工程师。
\end{itemize}

\section{\faHeartO\ 获奖情况}
\datedline{\textit{265/5204名}, 天池广告转化率预测比赛}{2018年5月}
\datedline{\textit{426/3152名}, 美年健康AI大赛-双高疾病风险预测}{2018年6月}
\datedline{\textit{第4/7名}, 猎豹移动黑客马拉松比赛}{2017年5月}
\datedline{猎豹移动OAT优秀技术支持}{2017}
\datedline{团队优秀奖}{2017}
\datedline{省级优秀信息化软件奖}{2014}

\section{\faInfo\ 其他}
% increase linespacing [parsep=0.5ex]
\begin{itemize}[parsep=0.5ex]
  \item 技术博客: http://www.cnblogs.com/accipiter/
  \item GitHub: https://github.com/AlexanLee
  \item 语言: 英语 - 熟练,无障碍与欧美、海外同事、客户沟通、解决问题。
  \item 业余爱好:坚持每周一次游泳。
\end{itemize}

%% Reference
%\newpage
%\bibliographystyle{IEEETran}
%\bibliography{mycite}
\end{document}
