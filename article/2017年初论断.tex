\documentclass{article}
\usepackage{ctex}

\title{5G时代的机遇}
\begin{document}
	5G时代即将到来。\par
	某种意义来说,是4G时代成就了现在的移动互联网,当然,移动设备的普及以及互联网的每一个从业者都起到了很大的作用,但是真正的能够让现在的移动互联网这么活跃,特别是在中国现在的情况下,上到几十岁的老人,下到小孩,无不沉浸其中,这其中也有些超级的APP,比如在社交领域的微信,其体量、活跃、渗透率没有能够与其匹敌的的产品。那么对于,下一代,或者说能够颠覆微信的东西到底会是什么东西了,对于国外社交领域有FB的成功,有Line的活跃,也有Whatsapp,Snapchat等的异军突起,但是,现在,在现有模式下要想做出一个产品颠覆这些巨头,是不可能的事情,那么,真正要做的事情,只有把这个时代淘汰掉,一旦这个时代淘汰掉了,那么这个时代的所有成功者也自然都被击败了,那么什么东西能够成为下一个时代颠覆性的东西了?或着说能够开启下一时代了?
	\par
	就目前看到的,我觉得视频社交会在语音社交之后一个大突破点。16年初期是各种直播平台爆发的时候,国内的如火如荼,双十一的时候,主播秀,助力各个卖家,也是取得了可喜的成绩,国际上反应稍微慢些,但是一些公司的产品一推出,就取得了很好的成绩,比如Live.me,这正说明,视频社交是有很好的用户环境的。
	\par
	16年还有一些新的,比较突出的技术那就VR,AR等。这些新的技术,在双十一的大购和春节的抢红包已经深入人心了。再有就是人工智能技术,深度学习的突破与应用,AlphaGo挑战人类棋手的赛事,得到了广泛的关注,之后,就是一轮又一轮的技术布局,和各个巨头产品的不断推出。大到汽车特斯拉的无人驾驶,小到谷歌的助理,亚马逊的Echo。当然,值得推崇的还有,或者说振奋人心的,是华为带来的5G时代的开启,未来的网络速度会更加的快,穿透力更加的强,那么,个人认为,这算是能开启一个时代的技术支点。
	\par
	在新的移动网络下,会带动物联网的发展,更加重要的是,个人认为,未来的社交会更加的注重体验式,就比人说,哥几个要交流,需要找一个小酒馆,小咖啡厅,小KTV一样,网络速度和质量的提升,对于视频的社交变得更加的方便和可能,那么,在上边说的一些技术的叠加效果下,假如,一个群里6个聊天对象,那么完全可以做成在夏威夷一起聊天的一个体验场景,这有赖于VR技术的落地,网络的可用性。还有一点,就是可以把熟人社交做到真正面对面。比如,两口子,相隔异地,相互交流,也可以和在一个家里一样。
	\par
	还有一点,值得分享一下,那就是以前的互联网靠着免费模式,得到爆发式的增长,而现在,一些付费的模式,也得到了很多的认可,时代的变化,商业模式也要同步跟进。
\end{document}